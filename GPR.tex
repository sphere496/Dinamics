\documentclass{jsarticle}
\usepackage{standard-jpn}
\usepackage{url}
\usepackage{listings,jlisting}
\lstset{%
  language={C},
  basicstyle={\small\ttfamily},%
  identifierstyle={\small},%
  commentstyle={\small\itshape},%
  keywordstyle={\small\bfseries},%
  ndkeywordstyle={\small},%
  stringstyle={\small\ttfamily},
  frame={tb},
  breaklines=true,
  columns=[l]{fullflexible},%
  numbers=left,%
  xrightmargin=0zw,%
  xleftmargin=3zw,%
  numberstyle={\scriptsize},%
  stepnumber=1,
  numbersep=1zw,%
  lineskip=-0.5ex%
}

\title{力学系による将来のパラメータ推定}
\author{前多啓一}

\begin{document}

\maketitle


\section{基本的な方針}
\begin{rem}[方針]
 以下の方針で予測を行う.与えられたデータは,$n$個の観測ポイントにおける関数$x:\R\to\R^n\quad t\mapsto(x_1,\cdots,x_n)$の等間隔$\tau$の時間$t_1,\cdots,t_m$でのデータである.推定するのは,$k$番目の特定の変数$x_k$の将来での動きである.
\begin{itemize}
 \item $(1,2,\cdots, n)$のなかから,$L$個の数が入っているタプルを$s$個選ぶ.
 \item $l$番目のタプルから,次の値を最小化する$\psi_l:\R^L\to\R$を推定する.(ガウス過程回帰)
       \[
	\sum_{i=1}^{m-1}|x_k(t_{i+1}) -\psi_l(x_{l_1}(t_i),x_{l_2}(t_i),\cdots,x_{l_L}(t_i))|
       \]
 \item 各$\psi_l$より1ステップの推定$\tilde{x}_k^l(t+\tau)=\psi_k^l(x_{l_1}(t),\cdots,x_{l_L}(t))$を計算する.
 \item 集めてできた推定の集合から,カーネル密度推定を行うことで,確率密度関数$p(x)$を推定する.
 \item 確率密度関数の歪度$\gamma$を計算し,$\gamma$が$0.5$以下であれば採用し,$\tilde{x}_k(t+\tau)=\int xp(x)dx$を推定として確定する.そうでなければ,以下のように推定値を修正する.交差検証によりインサンプルエラー$\delta_l$を計算し,それに従って$r$個のベストなサンプルを選び出す.
       \[
	\tilde{x}_k(t+\tau)=\sum_{i=1}^r\omega_i\tilde{x}_{k}^{l_i}(t+\tau)
       \]
       ここで,$\omega_{i}=\dfrac{\exp(-\delta_i/\delta_1)}{\sum_j\exp(-\delta_j/\delta_1)}$である.
\end{itemize} 
\end{rem}

\begin{defi}[カーネル密度推定]
$x_1,\cdots,x_n$を確率密度関数$f$をもつ独立同分布からの標本とする.カーネル関数$K$,バンド幅$h$のカーネル密度推定量とは,
\[
\hat{f}_h(x)=\dfrac{1}{nh}\sum_{i=1}^n K\left(\dfrac{x-x_i}{h}\right)
\]
基本的に,$K(x)=\dfrac{1}{\sqrt{2\pi}}e^{-x^2/2}$を使う.また,最適なバンド幅として,以下の値がある.
\[
h^*=\dfrac{c_1^{-2/5}c_2^{1/5}c_3^{-1/5}}{n^{1/5}},
\]
where $c_1=\ds\int x^2K(x)dx,\ c_2=\int K(x)^2dx,\ c_3=\int(f''(x))^2dx$.
\end{defi}

これについてはカーネル密度推定がscipyに標準搭載されているのでそちらを援用.

\section{ガウス過程回帰について}

Bishopを参照\cite{bishop}しながら,ガウス過程回帰について復習する.


\begin{itemize}
 \item 推定の仮定\\
       $y=\vw^T\phi(\vx)$とし,パラメータ$\vw$がガウス分布に従うと仮定する.\\
       すなわち,任意の$\vx_1,\cdots,\vx_n$に対し,$\vy=\Phi\vw$はガウス分布に従う.このことから,$\vy$は無限次元のガウス分布に従う,などとも言われる.ただし,$\Phi=(\phi(\vx_i))_{i=1,\cdots,n}$は計画行列である.
       % このとき,$y$の同時分布は,平均$0$共分散にグラム行列$K$があるガウス分布である.
       % \[
       % 	p(\vy)=N(\vy|0,K)
       % \]
       % ただし,$K=(k_{i,j})_ij$で,$k$はカーネル関数
 \item 与えられるデータ(サンプル)\\
       $\vx_1,\cdots,\vx_n\in\R^n$および$t_1,\cdots,t_n\in \R$
       ただし,$t_n=y_n+\varepsilon_n$であるとする.$\varepsilon_n$はノイズで,ガウス分布に従うとする.
 \item 推定するもの\\
       新しい入力$x_{n+1}$が与えられたときの出力$t_{n+1}$の確率分布を推定する.すなわち,
       \[
	p(t_{n+1}|\vx_{n+1},\vx_1,\cdots,\vx_n,t_1,\cdots,t_n)=N(t_{n+1}|m,\sigma^2)
       \]
       における$m$と$\sigma^2$を推定する.
\end{itemize}

\begin{thm}
 以下のようにカーネル関数のグラム行列を定義する.
 \[
  K=(k(\vx_i,\vx_j))_{i,j}
 \]
 さらに,以下のように置く.
 \[
  \vt=\begin{pmatrix}
       t_1\\\vdots\\t_n
      \end{pmatrix},\quad
 \vk=\begin{pmatrix}
	k(\vx_1,\vx)\\\vdots\\k(\vx_n,\vx)
     \end{pmatrix}
 \]
 最適な推定は,以下の通り.
 \begin{align*}
  m&=\vk^T(K+\sigma_n^2I)\inv \vt\\
  \sigma^2&=k(\vx,\vx)-\vk^T(K+\sigma_n^2I)\vk
 \end{align*}
\end{thm}

\section{コード}

以上を踏まえ,以下のようにコードを組んだ(参考\cite{GPY})

\lstinputlisting[caption=main.py,label=ほげ]{main.py}

\section{結果}

以下の同業種の株で推定を行った.\\

推定グループ(similar):業種グループ:Engineering\quad 業種サブグループ:Building\&Construc-Misc


\begin{center} 
'1801 JT Equity','1802 JT Equity','1803 JT Equity','1812 JT Equity','1820 JT Equity',\\
 '1821 JT Equity','1824 JT Equity','1833 JT Equity','1860 JT Equity','1893 JT Equity'
\end{center}

推定した株(terget):'1812 JT Equity'

推定に費やした回数(n):10回

推定に使った株の数(m):3ずつ

1回のカーネル密度推定は以下のようになる.

\begin{figure}[htb]
 \begin{center}
 \includegraphics[width = 10cm]{test_10_3.png}
 \caption{カーネル密度推定}
 \end{center}
\end{figure} 

以下は,10回すべての株を推定し,将来予測を行なった結果である. 

\begin{figure}[htb]
 \begin{center}
 \includegraphics[width = 10cm]{test.png}
 \caption{10回予測}
 \end{center}
\end{figure} 


\begin{thebibliography}{9}
 \bibitem{GPY}{Qiita PRML第6章 ガウス過程による回帰Python実装 \url{https://qiita.com/ctgk/items/4c4607edf15072cddc46}}
 \bibitem{bishop} Christopher M. Bishop ``Pattern Recognition and Machine Learning'' 2013	  
\end{thebibliography}


\end{document}

